% !TEX program = xelatex
\documentclass[
    10pt,
    oneside,
    openany,
    b5paper,
    colorscheme = black  % 请根据需要选择或定制配色方案
]{qyxf-book}

\title{钱院学辅 \LaTeX 书籍模板}
\subtitle{\LaTeX{} book class of Qian Yuan Xue Fu}  % 可选
\author{钱小辅}
\date{2019 年 2 月 29 日}
%\typo{AlphaGo}  % 排版人员信息,选填

% 定制元信息
\org{\Large\textit{西安交通大学}\\\textsc{Xi'an Jiaotong University}}
\footorg{\textsc{Qian Yuan Xue Fu}}
\cover{\includegraphics[width=.6\textwidth]{qyxf-logo.pdf}}
\license{}  % 清空许可证信息

% 调整封面标题大小
\renewcommand{\titlefont}{\Huge\bfseries}
\renewcommand{\subtitlefont}{\LARGE\itshape}

\begin{document}

\maketitle

\tableofcontents

\chapter{模板功能介绍}

\section{模板概况}

本模板名为 \verb|qyxf-book|,意即「\emph{钱院学辅之书}」,是西安交通大学\emph{钱}学森书\emph{院学}业\emph{辅}导中心自编资料时所采用的 \LaTeX 书籍模板。

利用 \verb|qyxf-book| 模板,我们已经生成了如下一些作品,它们都发布于\href{https://qyxf.site}{钱院学辅信息站}上,点击链接即可在浏览器中预览、下载。因各份作品采用了不同版本的 \verb|qyxf-book| 复本,且有一些个性化的定制,故观感将略有差异。

\begin{tcolorbox}
\begin{description}
    \item [GRE备考指南] \url{https://qyxf.site/latest/GRE备考指南-v2.0.pdf}
    \item [军事理论教程] \url{https://qyxf.site/latest/军事理论教程.pdf}
    \item [分析力学笔记] \url{https://qyxf.site/latest/分析力学笔记-v1.0.pdf}
    \item [大学物理题解] \url{https://qyxf.site/latest/大物题解(上).pdf}
    \item [实变函数习题解答] \url{https://qyxf.site/latest/实变函数习题解答.pdf}
    \item [计算方法撷英] \url{https://qyxf.site/latest/计算方法撷英-v1.1.pdf}
    \item [计算机程序设计指南] \url{https://qyxf.site/latest/计算机设计程序指南.pdf} 
\end{description}
\end{tcolorbox}

根据钱院学辅所编资料的题材与传播形式,本套模板的设计宗旨确定为:
\begin{enumerate}
    \item 实现钱院学辅自编资料所需功能,同时留出供其他用户定制的接口;
    \item 保持页面样式的干净、整洁,不引入与作品内容的呈现无关的元素;
    \item 保证文档的打印效果,以及在手机、平板等小屏设备上的阅读效果。
\end{enumerate}

\section{版本演进}

至目前为止,\verb|qyxf-book| 已经发展了三代版本。当前模板的版本为 \textbf{v\styversion}。

\subsection{V1 版本}
最早的版本。在该版本中确定了模板当前的格局(封面、信息页等)。结构、样式上效仿于 \href{https://elegantlatex.org/}{Elegant\LaTeX} 系列模板,借用了其章节标题样式、配色选项,并定义了一套字体(中文采用 Adobe 系列字体,西文采用 Cambria + Calibri,数学字体采用 Cambria Math)。

因技术水平,存在诸多兼容性问题,使用多有不便。仅有 \textbf{v1.0} 一个正式版本。

\subsection{V2 版本}

完全摆脱 Elegant\LaTeX{} 的框架,从头独立编写。与笔记模板 \verb|qyxf-note|、告示模板 \verb|qyxf-notice|(均已不再开发)共三个模板合并为 \verb|qyxf-sets| 宏集,采用 doc/docstrip 套件一同开发;后因开发成本过高,\verb|qyxf-sets| 宏集停止开发,分离出来独立开发。

在 V1 的基础上趋于成熟,优化了字体、封面与章节标题样式,并初步定义了一些用户常用的环境(如数学定理)。利用 \verb|pgfornament| 重新设计了章节标题、页脚的样式,取消了配色(仅用灰度色系)。

目前钱院学辅出品的大多数资料采用此版本模板编写。

\begin{itemize}
  \item \textbf{v2.0}:独立开发的首个版本,确定了整个模板的框架。
  \item \textbf{v2.0a}:修正了说明文档的问题。
  \item \textbf{v2.1}:定义了可选项 \verb|sourcefont| 以启用思源字体为中文字体,将默认字体交还 C\TeX 宏集处理。
  \item \textbf{v2.2}(\verb|qyxf-sets|):实现了取消装饰的选项 \verb|nodecoration|;修正了字体设置卡顿的问题。
  \item \textbf{v2.2}(\verb|qyxf-book|):分离出来之后采用同一版本号再发布一次,增加了 demo 文件。
  \item \textbf{v2.3}:由 \href{https://github.com/SciZeal}{SciZeal} 补充了 demo 中的段落示例、\hologo{BibTeX} 示例等内容。
\end{itemize}

\subsection{V3 版本}

当前版本。在 V2 版本之上进一步改进了之前的问题,并向通用、可定制的方向前进一大步。

在字体方面,移除了中文字体备选项,完全采用 C\TeX 宏集的默认设置;西文及数学字体由 \verb|newtx| 系列宏包设定,由此使模板本身完全摆脱对外部字体的依赖\footnote{由此也使得字体的定制成为用户自己的工作。}。

在样式方面,综合若干书籍及网页模板重设了各类元素的尺寸、间距,并完全移除了原由 \verb|pgfornament| 与 \verb|pgfornament-han| 宏包所实现的装饰\footnote{一方面,是因为 \TeX Live 2020 中这两个宏包未被容纳进来;另一方面,目前也发现装饰的存在不利于样式的修改、定制,且对阅读效果并无明显改善。}。此外,还新增了用户定制配色方案、元信息的功能。

\subsection{我该用哪一个版本?}

若需要编译由钱院学辅 2020 年 8 月前发布的资料之 \LaTeX 源码,请直接用源码中附带的 \verb|qyxf-book.cls| 文件编译,或下载 V2 系列最新版本 v2.3\footnote{链接:\url{https://gitee.com/qyxf/qyxf-book/releases/2.3}}。否则,请采用最新版本的模板套装\footnote{包括模板文件 \texttt{qyxf-book.cls} 及图片、配色方案等附件。}。

\section{功能说明}

以下简要说明 \verb|qyxf-book| 提供的功能及其使用方法。

\subsection{命令与环境}

本模板提供了以下几个引导命令,主要用于题解、教辅的编写。这些命令都自带 \verb|\noindent| 控制序列,请放在段首使用。

\begin{itemize}
  \item \verb|\exercise{<id>}|:生成一个「练习」标记,接受一个参数 \verb|<id>| 作为编号
  \footnote{模板中没有为这个命令设定自动编号,主要是因为编写题解的过程中常出现选题、跳题的情况,且自动编号下不易从源代码中找到对应题目。}
  (可留空),如 \verb|\exercise{1}| 将生成:\exercise{1}。
  \item \verb|\solve|:生成一个「解」的标记:\solve。
  \item \verb|\analysis|:生成一个「分析」的标记:\analysis。
\end{itemize}

除此之外,还提供了常见的用户环境,包括:
\begin{itemize}
  \item 定理类环境:定理 \verb|theorem|、引理 \verb|lemma|
  \item 定义类环境:定义 \verb|definition|
  \item 其他环境:注记 \verb|note|、警告 \verb|alert|
\end{itemize}
其中,定理、定义类环境属于数学环境,采用 \LaTeX 原生机制定义,仅在外围包裹一层装饰用的盒子;因此,可按照通常的写法编写数学环境,如:
\begin{tcolorbox}
\begin{verbatim}
\begin{theorem}[L'H\^opital 法则]
\begin{equation}
\lim_{x\to x_0} \frac{f(x)}{F(x)} =
\lim_{x\to x_0} \frac{f'(x)}{F'(x)}.
\end{equation}
\end{theorem}
\end{verbatim}
\end{tcolorbox}
将生成:
\begin{theorem}[L'H\^opital 法则]
\begin{equation}
\lim_{x\to x_0} \frac{f(x)}{F(x)} =
\lim_{x\to x_0} \frac{f'(x)}{F'(x)}.
\end{equation}
\end{theorem}

\subsection{配色方案定制}

本模板支持用户自定义的配色方案,写法请参考发布版本中 \verb|colors| 目录下的文件。目前随模板一同发布的配色方案包括:
\begin{itemize}
  \item \verb|basic|:基本的灰度配色方案,仿效于 \verb|qyxf-book| V2 版本的设计,默认值;
  \item \verb|bootstrap-v4|:借鉴 BootStrap V4 中的配色方案;
  \item \verb|black|:只有黑白两色,非常适合改造成其他的双色配色方案。
  \item \verb|rbb| 红色 + 蓝色 + 黑色(Red Blue Black),在屏幕上比较醒目。
\end{itemize}

配色方案定制完成后,请在文档类中通过键值对的形式引用,如
\begin{tcolorbox}
\verb|colorsheme = rbb| 
\end{tcolorbox}
将启用 \verb|rbb| 配色方案。

配色方案文件请\textbf{务必}放在待编译文件所在目录的 \verb|colors| 子目录下。关于配色方案原理、定制方法的更多说明,将在之后版本完善。

\subsection{元信息定制}

本模板中要求填写几项元信息,如表 \ref{tab:metadata} 所示。请在使用时填写完整,以避免出现预料之外的情况。

\begin{table}[htbp]
\centering\small
\caption{\texttt{qyxf-book} 中需求的元信息}
\label{tab:metadata}
\begin{tabular}{cccc}
\toprule
元信息 & 设置命令 & 可否置空 & 默认值 \\
\midrule
标题 & \verb|\title| & \ding{55} & 无 \\
副标题 & \verb|\subtitle| & \ding{55} & 无\\
作者 & \verb|\author| & \ding{55} & 无\\
创作日期 & \verb|\date| & \ding{55} & \verb|\today|\\
排版者 & \verb|\typo| & \ding{51} & 无 \\
\midrule
首页组织信息 & \verb|\org| & \ding{51} & \begin{tabular}[c]{@{}l@{}}
\verb|\textit{钱学森书院学业辅导中心}\\[1ex]|\\\verb|\textsc{Xi'an Jiaotong University}|\end{tabular}\\
\midrule
页脚组织信息 & \verb|\footorg| & \ding{51} & \verb|\textsc{Qian Yuan Xue Fu}| \\
\midrule
许可证说明 & \verb|\license| & \ding{51} & \begin{tabular}[c]{@{}l@{}}
\verb|本作品采用\href{https://|\\
\verb|creativecommons.org/licenses/|\\
\verb|by-nc-nd/4.0/}{\bfseriesCC BY-|\\
\verb|NC-ND 4.0 协议}进行许可。使用|\\
\verb|者可以在给出作者署名及资料来源|\\
\verb|的前提下对本作品进行转载,但不|\\
\verb|得对本作品进行修改,亦不得基于|\\
\verb|本作品进行二次创作,不得将本作|\\
\verb|品运用于商业用途。|
\end{tabular} \\
\bottomrule
\end{tabular}
\end{table}

\subsection{封面定制}

本模板提供了简易的封面定制接口,通过 \verb|\cover| 命令来填写封面上需要的内容。例如,需要将本模板中自带的 \verb|cover.pdf| 文件插入到封面中心,通过以下命令即可:

\begin{tcolorbox}
\begin{verbatim}
\cover{
  \includegraphics[width=0.5\textwidth]{figure/cover.pdf}
}
\end{verbatim}
\end{tcolorbox}
若你希望使插入元素靠向上侧,则可这样实现:
\begin{tcolorbox}
\begin{verbatim}
\cover{
  \includegraphics[width=0.5\textwidth]{figure/cover.pdf}
  \vfill
}
\end{verbatim}
\end{tcolorbox}
后侧的 \verb|\vfill| 会将插入元素向上挤压。在另一方向上同理。除了插入图片之外,你也可以使用 \verb|\cover| 命令在首页上插入全局性命令,或使用 Ti$k$Z 环境按页面定位插入图片。 

此外,封面标题的字体大小、样式也可重新定义,字体接口及默认值如表 \ref{tab:preface-font} 所示。如有需要(如标题名称较长导致换行),请使用 \verb|\renewcommand| 命令修改。

\begin{table}[htbp]
\centering
\caption{\texttt{qyxf-book} 提供的封面字体修改接口}
\label{tab:preface-font}
\begin{tabular}{ccc}
\toprule
接口 & 对应元信息 & 默认值\\
\midrule
\verb|\titlefont| & \verb|\title| & \verb|\Huge\bfseries| \\
\verb|\subtitlefont| & \verb|\subtitle| & \verb|\huge\itshape| \\
\verb|\authorfont| & \verb|\author| & \verb|\LARGE\itshape| \\
\verb|\datefont| & \verb|\date| & \verb|\Large| \\
\bottomrule
\end{tabular}
\end{table}

\subsection{正文字体设定}

西文字体方面,目前采用 \verb|newtx| 系列宏包设定。如您想变更西文字体,请完成以下两个步骤:

\begin{enumerate}
    \item 为 \verb|qyxf-book| 文类传入 \verb|newtx = false| 的选项,此时西文字体将还原至 Computer Modern 字体;
    \item 采用 \verb|fontspec| 宏包提供的 \verb|\setmainfont| 等命令设定字体。
\end{enumerate}

例如,下面的代码会将西文字体调整为 \TeX 发行版自带的 Libertinus 系列:

\begin{tcolorbox}
\begin{verbatim}
\documentclass[newtx = false]{qyxf-book}
% fontspec 宏包已载入,无需再声明
\setmainfont{Libertinus Serif}
\setsansfont{Libertinus Sans}
\setmonofont{Libertinus Mono}
\end{verbatim}
\end{tcolorbox}

中文字体方面,目前完全交由 \verb|ctex| 宏集选取默认字体。如你想变更中文字体,请直接在文档导言区采用 \verb|xeCJK| 宏包提供的 \verb|\setCJKmainfont| 等命令设置。例如,下面的代码会将中文字体调整为思源字体(采用伪斜体)
\footnote{请确保您所使用的字体正确安装。}:
\begin{tcolorbox}
\begin{verbatim}
\setCJKmainfont[FakeItalic]{Source Han Serif SC}
\setCJKsansfont[AutoFakeSlant]{Source Han Sans SC}
\setCJKmonofont[AutoFakeSlant]{Source Han Sans SC}
\end{verbatim}
\end{tcolorbox}
采用此种方式设定字体会触发若干警告,但对编译过程无影响。因 \verb|ctex| 相关机制,目前没有很好办法消除这一问题。

\section{问题及改进}

\subsection{已知问题}

本模板目前还存在许多可改进之处,主要包括:
\begin{enumerate}
    \item 模板对 \verb|part| 级别的标题、目录样式未作任何定制;
    \item 在参考文献方面,模板仅对 \LaTeX 自带的简易文献环境 \verb|thebibliorgraphy| 做了定制,未考虑 \hologo{BibTeX}
    \footnote{在 v2.3 版本中曾在 demo 中引入了 \hologo{BibTeX} 的示例;后考虑到此方面并无需求,为了简化 demo 结构,在 V3 版本中又移除了。}
    及 \verb|biblatex| 的样式支持;
    \item 模板尚未引入对代码抄录宏包(如 \verb|listings|)的支持;
    \item 配色方案还不完善,对部分元素的支持不佳。
\end{enumerate}
未做改动的原因很多,最大的原因是在钱院学辅排版的过程中对这些功能没有需求
\footnote{例如,在参考文献方面,许多资料甚至连 \texttt{thebibliorgraphy} 环境也不需要,用脚注就能解决问题。}。

\subsection{帮帮我们!}

如果您:
\begin{itemize}
    \item 对上述功能有需求,并有能力帮助我们完善;
    \item 有其他好的建议、功能推荐;
    \item 设计了新颖、实用的配色、字体方案;
\end{itemize}
请通过我们的代码仓库向我们提交建议与改动请求:\url{https://gitee.com/qyxf/qyxf-book},我们将酌情采纳、接收。

同时,我们也欢迎您透过 fork 的形式修改出适合您自己的模板,分发给更多人使用。本模板按 MIT 许可证发行,据此您享有充分的自由。这也是对于我们模板的一种推广
\footnote{模板滞销,帮帮我们!}
与帮助!

本模板目前由\textbf{黑山雁}维护,你也可以通过邮箱联系:\url{yjr134@163.com}。

\chapter{内容示例}
\section{用户环境示例}

\begin{define}
    极限就是超越自我。
\end{define}

\begin{theorem}
    任何极限都可以直接观察得出。
\end{theorem}

\begin{lemma}
    以上内容,纯属扯淡。
\end{lemma}

\begin{note}
好好学习,天天向上。
\end{note}

\begin{alert}
今天你学习了吗?
\end{alert}

\section{列表样式}
\begin{itemize}
    \item 这是第一层
    \item 这也是第一层
    \begin{itemize}
        \item 这是第二层
        \begin{itemize}
            \item 这是第三层
        \end{itemize}
    \end{itemize}
\end{itemize}

\begin{enumerate}
    \item 这是第一层
    \item 这也是第一层
    \begin{enumerate}
        \item 这是第二层
        \begin{enumerate}
            \item 这是第三层
        \end{enumerate}
    \end{enumerate}
\end{enumerate}

\section{正文示例}

\textbf{微分学}(\emph{differential calculus})是微积分的一部分,是通过\emph{导数}和\emph{微分}来研究曲线斜率、加速度、最大值和最小值的一门学科,也是探讨特定数量变化速率的学科。微分学是微积分的两个主要分支之一,另一个分支则是\textbf{积分学},探讨曲线下的面积。


\begin{table}[htbp]
\centering
\caption{常用导数}
\begin{tabular}{cccc}
\toprule
\textbf{原函数} & \textbf{导函数} & \textbf{原函数} & \textbf{导函数} \\
\midrule
$C$ & $0$ & $\ln x$ & $\frac{1}{x}$ \\
$x^\mu$ & $\mu x^{\mu - 1}$ & $\sin x$ & $\cos x$ \\
$e^x$ & $e^x$ & $\cos x$ & $-\sin x$ \\
\bottomrule
\end{tabular}
\end{table}

……几乎所有量化的学科中都有\textbf{微分}的应用。例如在物理学中,运动物体其\emph{位移}对时间的导数即为其\emph{速度},\emph{速度}对时间的导数就是\emph{加速度}、物体\emph{动量}对时间的导数即为物体所受的\emph{力},重新整理后可以得到牛顿第二运动定律 $F=ma$ 。化学反应的\emph{化学反应速率}也是导数。在运筹学中,会透过导数决定在运输或是设计上最有效率的做法。

\begin{figure}[htbp]
\centering
\includegraphics[width=.3\textwidth]{cover.pdf}
\caption{V2 版本的封面图片}
\label{fig:qyxf-logo}
\end{figure}

导数常用来找函数的\emph{极值}。含有微分项的方程式称为\textbf{微分方程},是自然现象描述的基础。微分以及其广义概念出现在许多数学领域中,例如\emph{复分析}、\emph{泛函分析}、\emph{微分几何}、\emph{测度}及\emph{抽象代数}\footnote{以上内容摘自维基百科中文词条 --- 微分学:\url{https://zh.wikipedia.org/wiki/微分学}。}。

\section{引导命令示例}

\exercise{1} 试用配方法求解方程:
\begin{equation}\label{eq:quadratic}
ax^2 + bx + c = 0
\end{equation}

\solve 首先,方程左右两侧同除以 $a$,得到
\[ x^2 + \frac bax + \frac ca = 0 \]
根据一次项来配方,按公式 $(x+A)^2=x^2+2Ax+A^2$ 配出常数项:
\[ x^2 + \frac bax + \left(\frac b{2a}\right)^2 + \frac ca - \left(\frac b{2a}\right)^2 = 0 \]
配方并移项得到
\[ \left(x + \frac b{2a}\right)^2 = \frac {b^2}{4a^2} - \frac ca \]
方程左右开方,得
\[ x + \frac b{2a} = \pm \sqrt{\frac {b^2}{4a^2} - \frac ca} \]
从而得到方程 \eqref{eq:quadratic} 之解为
\begin{equation}
x = - \frac b{2a} \pm \sqrt{\frac {b^2}{4a^2} - \frac ca}
\end{equation}
该式即为一元二次方程的\textbf{通用求根公式}。


\analysis 在这一问题中,需要注意以下几点 \cite{texbook,latex}:
\begin{itemize}
    \item ……
    \item ……
    \item ……
\end{itemize}

\begin{thebibliography}{99}
\bibitem{texbook} KNUTH~D~E. The \TeX book [M]. Addison-Wesley: Massachusetts, 1986.
\bibitem{latex} 刘海洋. \LaTeX 入门 [M]. 人民邮电出版社: 北京, 2013.
\end{thebibliography}

\end{document}