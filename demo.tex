% !TEX program = xelatex
\documentclass[
  10pt,
  twoside,
  openany,
  b5paper, % 以上均为 ctexbook 提供的文类选项
  colorscheme = basic, % 请根据需要选择或定制配色方案
]{qyxf-book}

\title{钱院学辅 \LaTeX 书籍模板}
\subtitle{\LaTeX{} book class of Qian Yuan Xue Fu}  % 可选
\author{钱小辅}
\date{2019 年 2 月 29 日}
%\typo{AlphaGo}  % 排版人员信息,选填

% 定制元信息
\org{\Large\textit{西安交通大学}\\\textsc{Xi'an Jiaotong University}}
\footorg{\textsc{Qian Yuan Xue Fu}}
\cover{\includegraphics[width=.6\textwidth]{qyxf-logo.pdf}}
\license{}  % 清空许可证信息

% 调整封面标题大小
\renewcommand{\titlefont}{\Huge\bfseries}
\renewcommand{\subtitlefont}{\LARGE\itshape}

\begin{document}

\maketitle

\tableofcontents

\chapter{模板功能介绍}

\section{模板概况}

本模板名为 \verb|qyxf-book|,意即「\emph{钱院学辅之书}」,是西安交通大学钱学森书院学业辅导中心(简称「\emph{钱院学辅}」)自编资料时所采用的 \LaTeX 书籍模板。

利用 \verb|qyxf-book| 模板,我们已经生成了如下一些作品,它们都发布于钱院学辅信息站
\footnote{网址:\url{https://qyxf.site},其中最新资料发布于 \url{https://qyxf.site/latest/} 页面上。}
上,点击以下各条链接即可在浏览器中预览、下载。因各份作品采用了不同版本的 \verb|qyxf-book| 复本,且有一些个性化的定制,故观感将略有差异。

\begin{tcolorbox}
  \begin{description}
    \item [GRE备考指南] \url{https://qyxf.site/latest/GRE备考指南-v2.0.pdf}
    \item [军事理论教程] \url{https://qyxf.site/latest/军事理论教程.pdf}
    \item [分析力学笔记] \url{https://qyxf.site/latest/分析力学笔记-v1.0.pdf}
    \item [大学物理题解] \url{https://qyxf.site/latest/大物题解(上).pdf}
    \item [实变函数习题解答] \url{https://qyxf.site/latest/实变函数习题解答.pdf}
    \item [计算方法撷英] \url{https://qyxf.site/latest/计算方法撷英-v1.1.pdf}
    \item [计算机程序设计指南] \url{https://qyxf.site/latest/计算机设计程序指南.pdf}
  \end{description}
\end{tcolorbox}

根据钱院学辅所编资料的题材与传播形式,本套模板的设计宗旨确定为:
\begin{enumerate}
  \item 实现钱院学辅自编资料所需功能,同时留出供其他用户定制的接口;
  \item 保持页面样式的干净、整洁,不引入与作品内容的呈现无关的元素;
  \item 保证文档的打印效果,以及在手机、平板等小屏设备上的阅读效果。
\end{enumerate}

\section{版本演进}

至目前为止,\verb|qyxf-book| 已经发展了三代版本。当前模板的版本为 \textbf{v\styversion}。

\subsection{V1 版本}
最早的版本。在该版本中确定了模板当前的格局(封面、信息页等)。结构、样式上效仿于 \href{https://elegantlatex.org/}{Elegant\LaTeX} 系列模板,借用了其章节标题样式、配色选项,并定义了一套字体(中文采用 Adobe 系列字体,西文采用 Cambria + Calibri,数学字体采用 Cambria Math)。

因技术水平,存在诸多兼容性问题,使用多有不便。仅有 \textbf{v1.0} 一个正式版本。

\subsection{V2 版本}

完全摆脱 Elegant\LaTeX{} 的框架,从头独立编写。与笔记模板 \verb|qyxf-note|、告示模板 \verb|qyxf-notice|(均已不再开发)共三个模板合并为 \verb|qyxf-sets| 宏集
\footnote{代码仓库地址:\url{https://github.com/qyxf/qyxf-sets/}(已经存档,无法编辑)。}
,采用 doc/docstrip 套件一同开发;后因这种方式的维护难度较大,\verb|qyxf-sets| 宏集停止开发,\verb|qyxf-book| 又分离出来独立开发。

在 V1 的基础上趋于成熟,优化了字体、封面与章节标题样式,并初步定义了一些用户常用的环境(如数学定理)。利用 \verb|pgfornament| 重新设计了章节标题、页脚的样式,取消了配色(仅用灰度色系)。

目前钱院学辅出品的大多数资料采用这一代模板编写。

\begin{itemize}
  \item \textbf{v2.0}:独立开发的首个版本,确定了整个模板的框架。
  \item \textbf{v2.0a}:修正了说明文档的问题。
  \item \textbf{v2.1}:定义了可选项 \verb|sourcefont| 以启用思源字体为中文字体,将默认字体交还 C\TeX 宏集处理。
  \item \textbf{v2.2}(\verb|qyxf-sets|):实现了取消装饰的选项 \verb|nodecoration|;修正了字体设置卡顿的问题。
  \item \textbf{v2.2}(\verb|qyxf-book|):分离出来之后采用同一版本号再发布一次,增加了 demo 文件。
  \item \textbf{v2.3}:由 \href{https://github.com/SciZeal}{SciZeal} 补充了 demo 中的段落示例、\hologo{BibTeX} 示例等内容。
\end{itemize}

\subsection{V3 版本}

当前版本。在 V2 版本之上进一步改进了之前的问题,并向通用、可定制的方向前进一大步。

在字体方面,移除了中文字体备选项,完全采用 C\TeX 宏集的默认设置;西文及数学内容采用 \TeX 发行版附带的 XITS 字体,使模板进一步减少了对外部字体的依赖\footnote{字体的定制交由用户完成,本模板不参与其中。}。

在样式方面,综合若干书籍及网页模板重设了各类元素的尺寸、间距,并完全移除了原由 \verb|pgfornament| 与 \verb|pgfornament-han| 宏包所实现的装饰\footnote{一方面,是因为 \TeX Live 2020 中这两个宏包未被容纳进来;另一方面,目前也发现装饰的存在不利于样式的修改、定制,且对阅读效果并无明显改善。}。

此外,还新增了用户定制配色方案、元信息的功能,这将为钱院学辅之外的其他用户使用本模板提供方便。

\subsection{我该用哪一个版本?}

若需要编译由钱院学辅 2020 年 8 月前发布的资料之 \LaTeX 源码,请直接用源码中附带的 \verb|qyxf-book.cls| 文件编译,或下载 V2 系列最新版本 v2.3\footnote{链接:\url{https://gitee.com/qyxf/qyxf-book/releases/2.3}}。否则,请采用最新版本的模板文件。

\begin{alert}
  请采用 \hologo{XeLaTeX} 编译使用本模板,其他场合下不保证本模板的功能正常;模板未来的开发之中也不会考虑对其他 \LaTeX 引擎的支持
  \footnote{未来新开发出来的 \LaTeX 引擎可能会被考虑。}。
\end{alert}

\section{常用功能指南}

\subsection{命令与环境}

本模板提供了以下几个引导命令,主要用于题解、教辅的编写。这些命令都自带 \verb|\noindent| 控制序列,请放在段首使用。

\begin{itemize}
  \item \verb|\exercise{<id>}|:生成一个「练习」标记,接受一个参数 \verb|<id>| 作为编号
        \footnote{模板中没有为这个命令设置自动编号,主要是因为编写题解的过程中常出现选题、
        跳题的情况,且自动编号下不易从源代码中找到对应题目。}
        (可留空),如 \verb|\exercise{1}| 将生成:\exercise{1}。
  \item \verb|\solve|:生成一个「解」的标记:\solve。(其可以接受一个可选参数来替换「解」字
  \footnote{你可以通过这个特性定义自己需要的标记命令,下面的分析标记同理。})
  ,如 \verb|\solve[答]| 将生成:\solve[答]。
  \item \verb|\analysis|:生成一个「分析」的标记:\analysis。(其可以接受一个参数来替换「分析」二字,如 \verb|\analysis[注记]| 将生成:\analysis[注记]。)
\end{itemize}

除此之外,还提供了常见的用户环境,包括:
\begin{itemize}
  \item 定理类环境:定理 \verb|theorem|、引理 \verb|lemma|
  \item 定义类环境:定义 \verb|definition|
  \item 其他环境:注记 \verb|note|、警告 \verb|alert|
\end{itemize}
其中,定理、定义类环境属于数学类环境,采用 \LaTeX 原生机制定义,仅在外围包裹一层装饰用的盒子;因此,可按照通常的写法编写数学类环境,如:
\begin{tcolorbox}
  \begin{verbatim}
\begin{theorem}[L'H\^opital 法则]
\begin{equation}
\lim_{x\to x_0} \frac{f(x)}{F(x)} =
\lim_{x\to x_0} \frac{f'(x)}{F'(x)}.
\end{equation}
\end{theorem}
\end{verbatim}
\end{tcolorbox}
将生成:
\begin{theorem}[L'H\^opital 法则]
\begin{equation}
\lim_{x\to x_0} \frac{f(x)}{F(x)} =
\lim_{x\to x_0} \frac{f'(x)}{F'(x)}.
\end{equation}
\end{theorem}

\subsection{配色方案定制}

\subsubsection{预定义配色方案}

目前,模板中已预定义的配色方案包括:
\begin{itemize}
  \item \verb|basic|:默认的灰度配色方案,与 \verb|qyxf-book| V2 版本中的配色接近。
  \item \verb|black|:只有黑白两色的配色方案。可以很容易的调整为\emph{自定义双色方案}。
  \item \verb|rbb|:红色 + 蓝色 + 黑色(Red Blue Black)的配色方案。可以很容易的调整为\emph{自定义三色方案}。
  \item \verb|bootstrap-v4|:借鉴 BootStrap V4 中的配色方案。
\end{itemize}
欲使用以上的方案,请在文档类中通过键值对的形式引用,如
\begin{tcolorbox}
\verb|colorsheme = rbb| 
\end{tcolorbox}
将启用 \verb|rbb| 配色方案。

\subsubsection{双色\&三色方案快速生成}
为方便用户快速修改预定义配色方案,模板为双色方案 \verb|black| 和三色方案 \verb|rbb| 设计了两个文类选项:
\begin{itemize}
  \item \verb|primary|:主色,相当于 \verb|black| 中的黑色及 \verb|rbb| 中的蓝色;
  \item \verb|seconadry|:辅助色,只对 \verb|rbb| 方案生效,相当于 \verb|rbb| 中的红色。
\end{itemize}
若您想快速定义一个自己的双色或三色配色方案,可直接通过以上两个文类选项设置。例如,若您希望使用橙 + 黑的双色主题,可通过以下的文类选项实现:
\begin{tcolorbox}
\begin{verbatim}
\documentclass[
  colorscheme = black,  % 载入双色主题,此时主色为黑色
  primary = orange  % 将主色改为橙色
]{qyxf-book}
\end{verbatim}
\end{tcolorbox}
而如果您希望使用橙 + 紫 + 黑的三色主题,则可通过以下的文类选项实现:
\begin{tcolorbox}
\begin{verbatim}
  \documentclass[
    colorscheme = rbb,  % 载入三色主题
    primary = orange,  % 将主色改为橙色
    secondary = violet  % 将辅助色改为紫色
  ]{qyxf-book}
\end{verbatim}
\end{tcolorbox}

\begin{alert}
请采用 \verb|xcolor| 宏包所提供的混色语法(如 \verb|red!50!blue|)设置对应的值,其他写法目前尚不支持。下同。
\end{alert}

\subsubsection{自定义配色方案}

若您需要编写自己的配色方案,请通过 \verb|\renewcommand| 命令修改表 \ref{tab:color-scheme} 所示的各项接口,在预定义主题的基础上修改相应颜色值。
\begin{table}[htbp]
\centering\small
\caption{配色方案接口表}\label{tab:color-scheme}
\begin{tabular}{llcc}
\toprule
条目 & 接口宏 & \parbox{5em}{\centering \texttt{basic}\\方案默认值} & 备注 \\
\midrule
标题文字 & \verb|\TitleColor| & \verb|black| & \\
列表标记 & \verb|\ListColor| & \verb|black| & \\
图表标题文字 & \verb|\CaptionColor| & \verb|black| & \\
链接文字 & \verb|\LinkColor| & \verb|black| & \\
杂项文字 & \verb|\MiscColor| & \verb|black| & \parbox{7em}{页码、引用标记、引导标记等} \\
\midrule
盒子背景 & \verb|\BoxBackground| & \verb|white| & \parbox{7em}{\texttt{tcolorbox} 环境默认定义,下同} \\
盒子边框 & \verb|\BoxFrame| & \verb|black!75| & \\
盒子标题文字 & \verb|\BoxTitleColor| & \verb|white| & \\
盒子标题背景 & \verb|\BoxTitleBackground| & \verb|black!50| & \\
盒子文字 & \verb|\BoxColor| & \verb|black| & \\
警告盒子背景 & \verb|\AlertBackground| & \verb|white| & \parbox{7em}{对 \texttt{alert} 环境定义,下同} \\
警告盒子文字 & \verb|\AlertColor| & \verb|black| & \\
警告盒子边框 & \verb|\AlertFrame| & \verb|black| & \\
警告盒子标题背景 & \verb|\AlertTitleBackground| & \verb|black| & \\
警告盒子标题文字 & \verb|\AlertTitleColor| & \verb|white| & \\
\midrule
定义环境背景 & \verb|\DefineBackground| & \verb|white| & \parbox{7em}{对 \texttt{define} 环境定义,下同} \\
定义环境边框 & \verb|\DefineFrame| & \verb|black!40| & \\
定义环境文字 & \verb|\DefineColor| & \verb|black| & \\
定理环境背景 & \verb|\TheoremBackground| & \verb|black!10| & \parbox{7em}{对各定理类环境定义,下同} \\
定理环境边框 & \verb|\TheoremFrame| & \verb|black!80| & \\
定理环境文字 & \verb|\TheoremColor| & \verb|black| & \\
\bottomrule
\end{tabular}
\end{table}

需要注意的是,在重定义链接及盒子环境的配色时,可能因设置命令提前展开接口而无法生效;在此情况下,请使用 \verb|\selectcolor| 命令强制更新这些颜色的值。例如,要变更链接颜色 \verb|\LinkColor| 及盒子文字颜色 \verb|\BoxColor| 为 \verb|red!50|,请通过以下命令实现:

\begin{tcolorbox}
\begin{verbatim}
\renewcommand{\LinkColor}{red!50}
\renewcommand{\BoxColor}{red!50}
\selectcolor  % 更新颜色
\end{verbatim}
\end{tcolorbox}

\subsection{元信息定制}

本模板中要求填写几项元信息,如表 \ref{tab:metadata} 所示。请在使用时填写完整,以避免出现预料之外的情况。

\begin{table}[htbp]
  \centering\small
  \caption{\texttt{qyxf-book} 中需求的元信息}
  \label{tab:metadata}
  \begin{tabular}{llcl}
    \toprule
    元信息       & 设置命令                & 可否置空  & 默认值                     \\
    \midrule
    标题         & \verb|\title| & \ding{55} & 无                         \\
    副标题       & \verb|\subtitle| & \ding{51} & 无                         \\
    作者         & \verb|\author| & \ding{55} & 无                         \\
    创作日期     & \verb|\date| & \ding{55} & \verb|\today|    \\
    排版者       & \verb|\typo| & \ding{51} & 无                         \\
    \midrule
    首页组织信息 & \verb|\org| & \ding{51} & \footnotesize\begin{tabular}[c]{@{}l@{}}
      \verb|\textit{钱学森书院学业辅导中心}\\[1ex]| \\\verb|\textsc{Xi'an Jiaotong University}|\end{tabular} \\
    \midrule
    页脚组织信息 & \verb|\footorg| & \ding{51} & \verb|\textsc{Qian Yuan Xue Fu}|    \\
    \midrule
    许可证说明   & \verb|\license| & \ding{51} & \footnotesize\begin{tabular}[c]{@{}l@{}}
      \verb|本作品采用\href{https://| \\
      \verb|creativecommons.org/licenses/| \\
      \verb|by-nc-nd/4.0/}{\bfseriesCC BY-| \\
      \verb|NC-ND 4.0 协议}进行许可。使用| \\
      \verb|者可以在给出作者署名及资料来源| \\
      \verb|的前提下对本作品进行转载,但不| \\
      \verb|得对本作品进行修改,亦不得基于| \\
      \verb|本作品进行二次创作,不得将本作| \\
      \verb|品运用于商业用途。|
    \end{tabular} \\
    \bottomrule
  \end{tabular}
\end{table}

\subsection{封面定制}

本模板提供了简易的封面定制接口,通过 \verb|\cover| 命令来填写封面上需要的内容。例如,需要将本模板中自带的 \verb|qyxf-logo.pdf| 文件插入到封面中心,通过以下命令即可:

\begin{tcolorbox}
\begin{verbatim}
\cover{\includegraphics[width=0.5\textwidth]%
{figure/qyxf-logo.pdf}}
\end{verbatim}
\end{tcolorbox}
若您希望使插入元素靠向上侧,则可这样实现:
\begin{tcolorbox}
\begin{verbatim}
\cover{\includegraphics[width=0.5\textwidth]%
{figure/qyxf-logo.pdf}\vfill}
\end{verbatim}
\end{tcolorbox}
后侧的 \verb|\vfill| 会将插入元素向上挤压。在另一方向上同理。除了插入图片之外,您也可以使用 \verb|\cover| 命令在首页上插入全局性命令,或使用 Ti$k$Z 环境按页面定位插入图片。

此外,封面标题的字体大小、样式也可重新定义,字体接口及默认值如表 \ref{tab:preface-font} 所示。如有需要(如标题名称较长导致换行),请使用 \verb|\renewcommand| 命令修改。

\begin{table}[htbp]
  \centering
  \caption{\texttt{qyxf-book} 提供的封面字体修改接口}
  \label{tab:preface-font}
  \begin{tabular}{ccc}
    \toprule
    接口                    & 对应元信息              & 默认值                  \\
    \midrule
    \verb|\titlefont| & \verb|\title| & \verb|\Huge\bfseries| \\
    \verb|\subtitlefont| & \verb|\subtitle| & \verb|\huge\itshape| \\
    \verb|\authorfont| & \verb|\author| & \verb|\LARGE\itshape| \\
    \verb|\datefont| & \verb|\date| & \verb|\Large| \\
    \bottomrule
  \end{tabular}
\end{table}

\section{正文字体设置}

\subsection{字体设置方案的考虑}

不像其他功能完备的模板,\verb|qyxf-book| 没有提供除默认配置之外的其他「字体包」,仅给出了关闭默认字体配置的选项,其余工作完全交由用户决定
\footnote{对于大多数用户而言,需要做的决定就是「什么也不用做,采用默认配置即可」。}
、执行。对这一点,有两方面考虑:
\begin{enumerate}
  \item 钱院学辅的成员大多使用 Windows 系统,在其上为 \TeX 环境配置字体比较困难。大多数同学无法载入外部字体,使得之前版本中提供的字体配置选项相当「鸡肋」,几乎无人使用。
  \item 对于少部分有能力自行配置字体的用户而言,预先定义的字体方案则往往不合其意,不如完全置空、由用户自行完成。
\end{enumerate}
因此,在当前版本的 \verb|qyxf-book| 模板中,完全移除了预定义的字体选项。如您认为有必要、也有能力自行配置字体,请阅读以下内容;否则,请遵循默认的字体配置。

\subsection{西文字体设置}

西文字体方面,目前采用与 Times 系列字体相近的 XITS 系列字体
\footnote{无衬线体采用接近于 Helvetica 的 \TeX{} Gyre Heros。}
。如您想变更西文字体,请完成以下两个步骤:

\begin{enumerate}
  \item 为 \verb|qyxf-book| 文类传入 \verb|xits = false| 的选项,此时西文字体将还原至 Computer Modern 字体;
  \item 采用 \verb|fontspec| 宏包提供的 \verb|\setmainfont| 等命令设置字体。
\end{enumerate}

例如,下面的代码会将西文字体调整为 \TeX 发行版自带的 Libertinus 系列
\footnote{其中并未设置数学字体,这种情况下模板将采用默认的数学字体。若需要调整数学字体,可采用 \texttt{unicode-math} 宏包配置。}
:

\begin{tcolorbox}
\begin{verbatim}
\documentclass[xits = false]{qyxf-book}
% fontspec 宏包已载入,无需再声明
\setmainfont{Libertinus Serif}
\setsansfont{Libertinus Sans}
\setmonofont{Libertinus Mono}
\end{verbatim}
\end{tcolorbox}

\subsection{中文字体设置}

中文字体方面,目前交由 C\TeX 宏集按系统自动选择默认字体。若您希望变更中文字体,同样需要两个步骤:
\begin{enumerate}
  \item 为 \verb|qyxf-book| 文类传入 \verb|noCJKfont| 选项,此时 C\TeX 宏集将把模板的中文字体设置清空;
  \item 仅完成以上一步,文档是不能编译的,还需采用 C\TeX 宏集的接口来选择其他中文字体,或采用 \verb|xeCJK| 宏包提供的命令自行设置。
\end{enumerate}

C\TeX 宏集提供了几种预定义的字体设置,除系统字体之外还包括:
\begin{itemize}
  \item Fandol 字体 \verb|fandol|:包括宋、楷、黑、仿宋四套字体,在 \TeX 发行版中自带,无需安装;
  \item 方正字体 \verb|founder|:包括方正书宋、方正黑体、方正楷体、方正仿宋等若干套字体,需自行安装,配置稍复杂;
  \item Adobe 字体:包括宋、楷、黑、仿宋四套字体,需自行安装。
\end{itemize}
在正确配置相关字体的情况下,可直接采用 C\TeX 宏集提供的 \verb|\ctexset| 命令配置,如下面的命令将使得模板采用 Fandol 字体:
\begin{tcolorbox}
\begin{verbatim}
\documentclass[noCJKfont]{qyxf-book}
\ctexset{fontset = fandol}
\end{verbatim}
\end{tcolorbox}
从方便使用的角度来说,Fandol 字体无需额外配置,显示效果也好于 Windows 系统的默认字体配置,但有缺字现象;对于需要印刷的作品,方正系列字体效果较好,字体完备
\footnote{前提是:应安装对应于较完备的 GBK 字符集的字体文件。}
,更适宜。

除了直接调用 C\TeX 的配置外,也可采用由 \verb|xeCJK| 宏包
\footnote{该宏包已经包含在模板中,其命令可以直接使用,无需在您的文档中载入。}
提供的若干命令自行配置字体。例如,下列命令将使得模板采用思源系列中文字体
\footnote{同样的,请确保相关字体已正确配置。}:
\begin{tcolorbox}
\begin{verbatim}
\documentclass[noCJKfont]{qyxf-book}
\setCJKmainfont[AutoFakeSlant]{Source Han Serif SC} % 伪斜体
\setCJKsansfont[AutoFakeSlant]{Source Han Sans SC}
\setCJKmonofont[AutoFakeSlant]{Source Han Sans SC}
\end{verbatim}
\end{tcolorbox}
以上命令未配置与中文字体对应的字族(如 \verb|zhsong| 之类),在一般场合没有问题。若有需要,请参考 C\TeX 文档及 \verb|xeCJK| 宏包文档进行设置。

\section{问题及改进}

\subsection{已知问题}

本模板目前还存在许多可改进之处,主要包括:
\begin{enumerate}
  \item 模板对 \verb|part| 级别的标题、目录样式未作任何定制;
  \item 在参考文献方面,模板仅对 \LaTeX 自带的简易文献环境 \verb|thebibliorgraphy| 做了定制,未考虑 \hologo{BibTeX}
        \footnote{在 v2.3 版本中曾在 demo 中引入了 \hologo{BibTeX} 的示例;后考虑到此方面并无需求,为了简化 demo 结构,在 V3 版本中又移除了。}
        及 \verb|biblatex| 的样式支持;
  \item 模板尚未引入对代码抄录宏包(如 \verb|listings|)的支持;
  \item 配色方案还不完善,对部分元素的支持不佳。
\end{enumerate}
未做改动的原因很多,最大的原因是在钱院学辅排版的过程中对这些功能暂时没有需求
\footnote{例如,在参考文献方面,许多资料甚至连 \texttt{thebibliorgraphy} 环境也不需要,用脚注就能解决问题。}。

\subsection{帮帮我们!}

如果您:
\begin{itemize}
  \item 对上述功能有需求,并有能力帮助我们完善;
  \item 有其他好的建议、功能推荐;
  \item 设计了新颖、实用的配色、字体方案;
\end{itemize}
请通过我们的代码仓库向我们提交建议与改动请求:\url{https://gitee.com/qyxf/qyxf-book},我们将酌情采纳、接收。

同时,我们也欢迎您透过 fork 的形式修改出适合您自己的模板,分发给更多人使用。本模板按 MIT 许可证发行,据此您享有充分的自由。这也是对于我们模板的一种推广
\footnote{模板滞销,帮帮我们!}
与帮助!

本模板目前由\textbf{黑山雁}维护,您也可以通过邮箱联系维护者:\url{yjr134@163.com}。

\subsection{可解决的问题}

在 \TeX\ Live 2019 中测试本模板时,发现 \textsc{Small Caps} 字族(\verb|\scfamily|)不能正常加载,回退到了普通字族;经过检查,发现默认采用的 XITS 字体在 \TeX\ Live 2019 中提供的版本(1.200)还未引入该字族,到 \TeX\ Live 2020 时对应的版本(1.300)方才完善了此部分。若您对于该字族有需求,请更新您的 \TeX 发行版。

\chapter{内容示例}
\section{用户环境示例}

\begin{define}
  极限就是超越自我。
\end{define}

\begin{theorem}
  任何极限都可以直接观察得出。
\end{theorem}

\begin{lemma}
  以上内容,纯属扯淡。
\end{lemma}

\begin{note}
  好好学习,天天向上。
\end{note}

\begin{alert}
  今天你学习了吗?
\end{alert}

\section{列表样式}
\begin{itemize}
  \item 这是第一层
  \item 这也是第一层
  \begin{itemize}
    \item 这是第二层
    \begin{itemize}
      \item 这是第三层
    \end{itemize}
  \end{itemize}
\end{itemize}

\begin{enumerate}
  \item 这是第一层
  \item 这也是第一层
  \begin{enumerate}
    \item 这是第二层
    \begin{enumerate}
      \item 这是第三层
    \end{enumerate}
  \end{enumerate}
\end{enumerate}

\section{正文示例}

\textbf{微分学}(\emph{differential calculus})是微积分的一部分,是通过\emph{导数}和\emph{微分}来研究曲线斜率、加速度、最大值和最小值的一门学科,也是探讨特定数量变化速率的学科。微分学是微积分的两个主要分支之一,另一个分支则是\textbf{积分学},探讨曲线下的面积。


\begin{table}[htbp]
  \centering
  \caption{常用导数}
  \begin{tabular}{cccc}
    \toprule
    \textbf{原函数} & \textbf{导函数}   & \textbf{原函数} & \textbf{导函数} \\
    \midrule
    $C$             & $0$               & $\ln x$         & $\frac{1}{x}$   \\
    $x^\mu$         & $\mu x^{\mu - 1}$ & $\sin x$        & $\cos x$        \\
    $e^x$           & $e^x$             & $\cos x$        & $-\sin x$       \\
    \bottomrule
  \end{tabular}
\end{table}

……几乎所有量化的学科中都有\textbf{微分}的应用。例如在物理学中,运动物体其\emph{位移}对时间的导数即为其\emph{速度},\emph{速度}对时间的导数就是\emph{加速度}、物体\emph{动量}对时间的导数即为物体所受的\emph{力},重新整理后可以得到牛顿第二运动定律 $F=ma$ 。化学反应的\emph{化学反应速率}也是导数。在运筹学中,会透过导数决定在运输或是设计上最有效率的做法。

\begin{figure}[htbp]
  \centering
  \includegraphics[width=.3\textwidth]{qyxf-cover.pdf}
  \caption{V2 版本的封面图片}
  \label{fig:qyxf-logo}
\end{figure}

导数常用来找函数的\emph{极值}。含有微分项的方程式称为\textbf{微分方程},是自然现象描述的基础。微分以及其广义概念出现在许多数学领域中,例如\emph{复分析}、\emph{泛函分析}、\emph{微分几何}、\emph{测度}及\emph{抽象代数}\footnote{以上内容摘自维基百科中文词条 --- 微分学:\url{https://zh.wikipedia.org/wiki/微分学}。}。

\section{引导命令示例}

\exercise{1} 试用配方法求解方程:
\begin{equation}\label{eq:quadratic}
  ax^2 + bx + c = 0
\end{equation}

\solve 首先,方程左右两侧同除以 $a$,得到
\[ x^2 + \frac bax + \frac ca = 0 \]
根据一次项来配方,按公式 $(x+A)^2=x^2+2Ax+A^2$ 配出常数项:
\[ x^2 + \frac bax + \left(\frac b{2a}\right)^2 + \frac ca - \left(\frac b{2a}\right)^2 = 0 \]
配方并移项得到
\[ \left(x + \frac b{2a}\right)^2 = \frac {b^2}{4a^2} - \frac ca \]
方程左右开方,得
\[ x + \frac b{2a} = \pm \sqrt{\frac {b^2}{4a^2} - \frac ca} \]
从而得到方程 \eqref{eq:quadratic} 之解为
\begin{equation}
  x = - \frac b{2a} \pm \sqrt{\frac {b^2}{4a^2} - \frac ca}
\end{equation}
该式即为一元二次方程的\textbf{通用求根公式}。


\analysis 在这一问题中,需要注意以下几点 \cite{texbook,latex}:
\begin{itemize}
  \item ……
  \item ……
  \item ……
\end{itemize}

\begin{thebibliography}{99}
  \bibitem{texbook} KNUTH~D~E. The \TeX book [M]. Addison-Wesley: Reading, 1986.
  \bibitem{latex} 刘海洋. \LaTeX 入门 [M]. 人民邮电出版社: 北京, 2013.
\end{thebibliography}

\end{document}